% -*- coding:UTF-8 -*-
%gougu.tex
%勾股定理
\documentclass[UTF8]{ctexart} %文档类

\usepackage{graphicx}
\usepackage{float}
\title{杂谈勾股定理}
\author{Tiketiskte}
\date{\today}

\bibliographystyle{plain} %声明文献参考格式
\newtheorem{thm}{定理}
%以上为导言区(preamble)

\begin{document} %正文 直接输出的部分

    \maketitle %输出论文标题
    \
    \tableofcontents %输出论文目录
    \section{勾股定理在古代} %开始新的一节
    西方称勾股定理为\emph{毕达哥拉斯定理}。将勾股定理的发现归功于公元前 6 世纪的
    毕达哥拉斯学派。该学派得到了一个法则,可以求出可排成直角三角形三边的三
    元数组。毕达哥拉斯学派没有书面著作,该定理的严格表述和证明则见于欧几里
    得\footnote{欧几里得,约公元前330--275年。}《几何原本》命题 47:“直角三角形斜边上的正方形等于两直角边上的两
    个正方形之和”。证明是用面积做的。

    我国《算经》载商高(约公元前 12 世纪)答周公问:
    \begin{quote}
        \zihao{-5}\kaishu
        勾广三,股修四,径隅五。
    \end{quote}
    又载陈子(约公元前 7--6 世纪)答荣方问:%--表示一个小短线 宽度与字母'n'相当
    \begin{quote}
        \zihao{-5}\kaishu
        若求邪至日者,以日下为勾,日高为股,勾股各自乘,并而开方除之,得邪至日。
    \end{quote}
    都较古希腊更早。......
    \section{勾股定理的近代形式}
    \begin{thm}[勾股定理]
        直角三角形斜边的平等于两腰的平方和。

        可以用符号语言表述为......
    \end{thm}
    \begin{equation}
        a(b+c) = ab + ac
    \end{equation}
    \begin{equation}
        \angle ACB = \pi / 2
    \end{equation}
    \begin{equation}
        AB^2 = BC^2 + AC^2.
    \end{equation}
    \begin{equation}
        AB_2 = BC_2 + AC_2;
    \end{equation}
    \begin{equation}
        90^\circ
    \end{equation}
    \begin{table}[H]
        \begin{tabular}{|rrr|}
            \hline
            直角边 $a$ & 直角边 $b$ & 斜边 $c$ \\
            \hline
                3 & 4 & 5 \\
                5 & 12 & 13 \\
            \hline
        \end{tabular}
        \qquad
        ($a^2 + b^2 = c^2$)
    \end{table}
    \begin{figure}[ht]
        \centering
        %\includegraphics[scale=0.6]{picture.jpg}
        \caption{这是祢豆子}
        \label{fig::p1}
    \end{figure}
    \bibliography{math}
\end{document}